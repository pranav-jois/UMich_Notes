\documentclass{article}
% Packages and document formatting
\usepackage[margin=1in]{geometry} 
\usepackage{amsmath, amsthm, amssymb, amsfonts, fancyhdr, color, comment, graphicx, 
            environ, dsfont, centernot, mathtools, xcolor, mathrsfs, mdframed, physics}
\usepackage{tikz,tkz-euclide,tikz-cd}
\setlength{\parindent}{0 pt}

% Document structural environments
\global\mdfdefinestyle{problemstyle}{%
    linecolor = white, linewidth = 0.75pt, topline = true, bottomline = true, backgroundcolor = black!, nobreak = true, fontcolor = white
}
\newenvironment{problem}[2][Problem]
    {\begin{mdframed}[style=problemstyle] \textbf{#1 #2}}{\end{mdframed}}
\newenvironment{solution}{\begin{proof}[Solution]}{\end{proof}}
\newenvironment{solution*}{\begin{proof}[Solution]}{\end{proof}}
    \AtBeginEnvironment{solution*}{\renewcommand{\qedsymbol}{}}


\newenvironment{lemma}[2][Lemma.]
    {\begin{mdframed}[style=problemstyle] \textbf{#1 #2}}{\end{mdframed}}
\newenvironment{example}[2][Example.]
    {\begin{mdframed}[style=problemstyle] \textbf{#1 #2}}{\end{mdframed}}
\newenvironment{definition}[2][Definition.]
    {\begin{mdframed}[style=problemstyle] \textbf{#1 #2}}{\end{mdframed}}
\newenvironment{theorem}[2][Theorem.]
    {\begin{mdframed}[style=problemstyle] \textbf{#1 #2}}{\end{mdframed}}
\newenvironment{proposition}[2][Proposition.]
    {\begin{mdframed}[style=problemstyle] \textbf{#1 #2}}{\end{mdframed}}
\newenvironment{corollary}[2][Corollary.]
    {\begin{mdframed}[style=problemstyle] \textbf{#1 #2}}{\end{mdframed}}


\newcommand{\remark}{\textbf{Remark. }}
\newcommand{\defn}{\textbf{Definition. }}


% Formatting commands
\newcommand{\jump}{\vspace{0.25 in}}
\newcommand{\newpar}{\vspace{12 pt}}
\newcommand{\nsjump}{\vspace{-4 pt}}
\newcommand{\nljump}{\vspace{-8 pt}}
\newcommand{\tab}{\hspace*{0.25 in}}
\newcommand{\stab}{\hspace*{0.1 in}}


% Math operators
\DeclareMathOperator{\id}{id}
\DeclareMathOperator{\image}{im}
\DeclareMathOperator{\Imag}{Imag}
\DeclareMathOperator{\nullity}{nullity}
\DeclareMathOperator{\FS}{FS}
\DeclareMathOperator{\sla}{\mathfrak{sl}}
\DeclareMathOperator{\ipstar}{IP^\star}
\DeclareMathOperator{\ipstarlim}{IP^\star lim}
\DeclareMathOperator{\ud}{UD}
\DeclareMathOperator{\smod}{mod}
\DeclareMathOperator{\plim}{plim}
\DeclareMathOperator{\im}{Im}
\DeclareMathOperator{\re}{Re}
\DeclareMathOperator{\dist}{dist}
\DeclareMathOperator{\Area}{Area}
\DeclareMathOperator{\Int}{Int}
\DeclareMathOperator{\Ext}{Ext}
\DeclareMathOperator{\law}{law}


% Frequently used characters
\newcommand{\R}{\mathbb{R}}  
\newcommand{\Z}{\mathbb{Z}}
\newcommand{\N}{\mathbb{N}}
\newcommand{\Q}{\mathbb{Q}}
\newcommand{\C}{\mathbb{C}}
\newcommand{\F}{\mathbb{F}}

\newcommand{\vi}{\mathbf{i}}
\newcommand{\vj}{\mathbf{j}}
\newcommand{\vk}{\mathbf{k}}
\newcommand{\vx}{\mathbf{x}}
\newcommand{\vy}{\mathbf{y}}
\newcommand{\vz}{\mathbf{z}}
\newcommand{\vn}{\mathbf{n}}
\newcommand{\vr}{\mathbf{r}}
\newcommand{\vF}{\mathbf{F}}
\newcommand{\vG}{\mathbf{G}}
\newcommand{\vH}{\mathbf{H}}
\newcommand{\vE}{\mathbf{E}}
\newcommand{\vB}{\mathbf{B}}
\newcommand{\vP}{\mathbf{P}}
\newcommand{\vM}{\mathbf{M}}
\newcommand{\vA}{\mathbf{A}}
\newcommand{\vL}{\mathbf{L}}


\newcommand{\vell}{\boldsymbol{\ell}}
\newcommand{\vtheta}{\boldsymbol{\theta}}
\newcommand{\vphi}{\boldsymbol{\phi}}

\newcommand{\rhat}{\hat{\mathbf{r}}}
\newcommand{\phihat}{\hat{\boldsymbol{\phi}}}
\newcommand{\thetahat}{\hat{\boldsymbol{\theta}}}
\newcommand{\zhat}{\hat{\mathbf{z}}}
\newcommand{\yhat}{\hat{\mathbf{y}}}
\newcommand{\xhat}{\hat{\mathbf{x}}}
\newcommand{\nhat}{\hat{\mathbf{n}}}

\newcommand{\as}{\text{ as }}
\newcommand{\so}{\text{ so }}

\newcommand{\scrA}{\mathscr{A}}
\newcommand{\scrB}{\mathscr{B}}
\newcommand{\scrF}{\mathscr{F}}
\newcommand{\calL}{\mathcal{L}}


% New mathematical commands
\newcommand{\transpose}{\intercal}
\newcommand{\Deltastar}{\ensuremath{\Delta\hspace{-0.25ex}^\star}}
\newcommand{\poorDensity}[1]{\ensuremath{D_{\hspace{-0.3ex}#1}}}
\newcommand{\dbar}{\overline{d}}


% Quantum mechanics
\newcommand{\bilin}[2]{\langle #1 | #2 \rangle}
\newcommand{\sand}[3]{\langle #1 | #2 | #3 \rangle}
\newcommand{\expec}[1]{\left\langle #1 \right\rangle}
\newcommand{\vect}[1]{\overrightarrow{#1}}

% Miscellaneous commands
\renewcommand{\v}{\vb}
\renewcommand{\t}{\text}
\newcommand{\scr}{\mathscr}


% Include to be able to use \cupdot and \bigcupdot
\makeatletter
\def\moverlay{\mathpalette\mov@rlay}
\def\mov@rlay#1#2{\leavevmode\vtop{%
   \baselineskip\z@skip \lineskiplimit-\maxdimen
   \ialign{\hfil$\m@th#1##$\hfil\cr#2\crcr}}}
\newcommand{\charfusion}[3][\mathord]{
    #1{\ifx#1\mathop\vphantom{#2}\fi
        \mathpalette\mov@rlay{#2\cr#3}
      }
    \ifx#1\mathop\expandafter\displaylimits\fi}
\makeatother
\newcommand{\cupdot}{\charfusion[\mathbin]{\cup}{\cdot}}
\newcommand{\bigcupdot}{\charfusion[\mathop]{\bigcup}{\cdot}}


% Changed old commands
\renewcommand{\emptyset}{\text{\O}}
\newcommand{\oldepsilon}{\epsilon}
\renewcommand{\epsilon}{\varepsilon}


% Graphics path
\graphicspath{{./figs}}

% invert color. change \mdfdefinestyle{problemstyle} command if submitting document
\pagecolor{black}
\color{white}

\title{Math 571 - Numerical Linear Algebra}
\author{Autumn 2024}

\begin{document}   

\maketitle

\section*{27 Aug 2024}
There are three types of problems that arise in numerical linear algebra.
\begin{itemize}
    \item[1.] Suppose $A \in \R^{n \times n}$ and $b \in \R^{n \times 1}$ is given and $A$ is invertible. Then solve $Ax = b$.
    \item[2.] Given $A \in \R^{n \times m}$ with $n > m$ and $b \in \R^n$, solve $Ax = b$.
    \item[3.] $A \in \C^{n \times n}$ is given. Find $\lambda \in \C$ and $x \in \C^n$, $x \neq 0$, such that $Ax = \lambda x$.  
\end{itemize}

There are direct and iterative methods. Sometimes $A$ is large, but sparse, and iterative methods make a little more sense. Problems of type 3 must be iterative by nature. 

\newpar

\begin{example} 
    Consider the temperature along a wire. $0 \leq x \leq 1$ with varying insulation thickness given by $u(x, t)$. We have $u_t = u_{xx} - f(x)$, with the initial condition $u(x, 0) = u_0(x)$. Furthermore, $u(0, t) = \alpha$ and $u(1, t) = \beta$. Is there a steady state temperature distribution along the wire?
    \newpar

    Consider the steady state $v(x)$ which must solve $v''(x) = f(x)$ and $v(0) = \alpha, v(1) = \beta$. To approximate, we divide the interval $[0, 1]$ into $N + 1$ units (grid spacing $h = 1/(N + 1)$) and solve the system $v_j = v(x_j)$. We need to deduce the unknowns $v_1$, $v_2$, $\hdots$, $v_N$. For $j \in \{2, 3, \hdots, N - 1\}$, we can use the approximation 
        \[
            v''(x_j) \approx \frac{v_{j + 1} - 2v_j + v_{j - 1}}{h^2}.
        \]
    Furthermore, for $j = 1$, we can note that 
        \[
            v''(x_1) \approx \frac{v_2 - 2 v_1 + \alpha}{h^2} = f_1
        \]
    and for $j = N$, we have
        \[
            v''(x_N) \approx \frac{\beta - 2v_N + v_{N - 1}}{h^2} = f_N.
        \]
    This is a linear system of $N$ equations in $N$ unknowns $v_1, \hdots, v_N$. This is alinear system of $N$ equations in $N$ unknowns $v_1, \hdots, v_N$:
        \[ 
            \begin{pmatrix}
                -2 & 1 & 0 & \hdots & 0 \\
                1 & -2 & 1 & \hdots & 0 \\
                \vdots & \vdots & \vdots & \vdots & \vdots \\
                0 & \hdots & 1 & -2 & 1 \\
                0 & \hdots & 0 & 1 & -2 
            \end{pmatrix}
            \begin{pmatrix}
            v_1 \\
            v_2 \\
            \vdots \\
            v_{N - 1} \\
            v_{N}
            \end{pmatrix} = 
            \begin{pmatrix}
                h^2 f_1 - \alpha \\
                h^2 f_2 \\
                \vdots \\
                h^2 f_{N - 1} \\
                h^2 f_N - \beta
            \end{pmatrix}
        \]
    This matrix is very sparse (fewer than 3$N$ nonzero entries) which are clustered along the diagonal. This is known as a \textit{tridiagonal} matrix. This system can be solved for given RHS in $O(n)$ operations using Gaussian elimination.
\end{example}

Fact: in the previous example, $M_N^{-1}$ is not sparse at all. In fact, it is almost completely dense. Consider an $N \times N$ matrix $A$ and multiply by an $N \times 1$ vector $w$. Then, computing $Aw$ involves $O(N^2)$ operations, so we will basically never form $A^{-1}$ when solving $Ax = b$. 
\newpar

We will look at Gaussian elimination, QR factorization, etc as direct methods, which we will interpret as matrix factorization. 
\newpar

A few simple facts:
\begin{itemize}[noitemsep]
    \item $A \in \R^{n \times m}$, $A: \R^m \mapsto \R^n$.
    \item $(Ax)_i = \sum_j A_{ij} x_j$.
    \item If $A_i$ refers to the $i$=th column of a matrix, then 
        \[Ax = x_1 A_1 + x_2 A_2 + \hdots + x_m A_m \]
\end{itemize}

\begin{definition}
    A \textit{norm} on $\C^n$ is a function $| \cdot | : \C^n \to \R^+$ satisfying:
    \begin{itemize}[noitemsep, nolistsep]
        \item[1.] $|x| \geq 0$ and $|x| = 0$ iff $x = 0$.
        \item[2.] For any $c \in \C$ and $x \in \C^n$, $|cx| = |c||x|$. 
        \item[3.] $|x + y| \leq |x| + |y|$.   
    \end{itemize}
\end{definition}
\begin{definition}
    We can consider the \textit{Euclidean norm} $|x|_2 = \sqrt{|x_1|^2 + \hdots + |x_n|^2}$. Furthermore, the \textit{p-norm} is defined by $|x|_p = \left(|x_1|^p + \hdots + |x_n|^p\right)^{1/p}$ for $1 \leq p \leq \infty$ ($p = 1$ is useful). Finally, $|x|_\infty = \max_{j = 1, \hdots, n} |x_j|$.  
\end{definition}

\begin{definition}
    Take $A \in \R^{n \times m}$. Then $||A||_F = \sqrt{\sum_{i, j} A_{i, j}^2}$. This is the \textit{Frobenius (Hilbert-Schmidt) norm}.
\end{definition}

\begin{definition}
    If $A \in \R^{n \times m}$, we can consider the \textit{induced p-norm} defined by considering 
        \[\max_{x \neq 0} \frac{|Ax|_p}{|x|_p} = \max_{|x| = 1}|Ax|_p.\]
\end{definition}

One can note that $p = 1$ and $ p = \infty$ are easier to work with than $p = 2$. First, let us consider the induced $p = 1$ norm of a matrix. Note that 
    \begin{align*}
        |Ax|_1 &= |A_1 x_1 + A_2 x_2 + \hdots + A_m x_m | \\
        &\leq |A_1 x_1| +  |A_2 x_2| + \hdots + |A_m x_m| \\
        &= |x_1| |A_1| + |x_2||A_2|+ \hdots + |x_m||A_m| \\
        &\leq |a_1| |A_{i_*}| + |a_2| |A_{i_*}| + \hdots + |a_n| |A_{i_*}|
    \end{align*}
where $|A_{i_*}| = \max_{1 \leq j \leq m} |A_j|$. Thus, 
    \[
        |Ax| \leq \underbrace{(|x_1| + \hdots + |x_m|)}_{\text{sum} = 1} |A_{i_*}| = |A_{i_*}|
    \]
so we get the bound $|A|_1 \leq |A_{i_*}|$. It is easy to see the upper bound is attained by chosing an appropriate standard basis vector, so it follows that $|A|_1 \leq |A_{i_*}|$. 

\section*{29 August 2024}
First, we can talk a little more about the Frobenius norm on $A \in \C^{n \times m}$. We can note that 
    \begin{align*}
        \left(\sum_{i, j} |A_{i,j}|^2 \right)^{1/2} = \sqrt{\sum_{i, j} \overline{A_{i, j}} A_{{i, j}}} = \sqrt{\sum_{i, j}(A^{*})_{j, i}A_{i, j}} = \sqrt{\sum_j(A^* A)_{j, j}} = \sqrt{\tr(A^*A)}.
    \end{align*}
Furthermore, we discussed the \textit{induced norm} defined on $1 \leq p \leq \infty$ defined by 
    \begin{align*}
        ||A||_p = \max_{|x|_p = 1} |Ax|_p = \max_{x \neq 0}.
    \end{align*}
Recall $|| \cdot ||_1$ and $|| \cdot ||_\infty$ are often the easiest induced matrix norms to compute. We have already seen how to compute the $|| \cdot ||_1$ norm so now we will attempt to compute $|| \cdot ||_\infty$. We have
    \begin{align*}
        |(Ax)_i| = |\sum_j A_{i, j} x_j| \leq \sum_j |A_{i, j}| |x_j| \leq \sum_j |A_{i, j}|.
    \end{align*}
This is equal to the absolute row sum along the row $i$. Thus, 
    \begin{align*}
        |Ax|_{\infty} = \max_i|(Ax)_i| \leq \max_i \sum_j |A_i, j|.
    \end{align*}
It is easy to find a value of $x$ such that the inequality turns into an equality.
\newpar

Finally, we can talk about the induced 2-norm on matrices. 
We have the following result:
\begin{theorem}
    If $A \in \C^{n \times m}$, then $|| A ||_2 = \sqrt{\lambda}$, where $\lambda$ is the largest eigenvalue of $A^* A$. 
\end{theorem}

\begin{proof}
Suppose $A \in \C^{n \times m}$. Then, by definition,
    \begin{align*}
        ||A||^2_2 = \max_{|x|_2 = 1} |Ax|^2_2,
    \end{align*}
and we can write $|Ax|_2^2 = \langle Ax, Ax\rangle = \langle A^* A x, x \rangle$ under the usual inner product. Thus, our optimization problem turns into 
    \begin{align*}
        ||A||_2^2 = \max_{|x|_2 = 1} \langle A^* A x, x \rangle
    \end{align*}
where $A^* A \in \C^{m \times M}$. Now, $A^* A$ is Hermitian, so $A^*A$ is \textit{unitarily diagonizable} (eigenvectors can be chosen to form an orthonormal basis). Moreover, all eigenvalues are real and positive. Take $\{v_1, \hdots, v_m\}$ to be the orthonormal basis satisfying $A^*A v_j = \lambda j v_j$. If we take $V = (v_1^T, \hdots, v_n^T)$ and 
    \begin{align*}
        \Sigma = \begin{pmatrix}
        \lambda_1 & 0 & \hdots & 0 \\
        0 & \lambda_2 & \hdots & 0 \\
        \vdots & \vdots & \vdots & \vdots \\
        0 & 0 & \hdots & \lambda_m
        \end{pmatrix},
    \end{align*}
then $A^* A = V \Sigma V^*$. Thus, 
\begin{align*}
    |Ax|_2^2 &= \max_{|x|_2 = 1} \langle A^* Ax, x\rangle = \max_{|x|_2 = 1} \langle \Sigma V^* x, V^*x \rangle.
\end{align*} 

Since $V^* = V^{-1}$, it follows that 
    \begin{align*}
        \langle V^* x, V^*x\rangle = \langle x, VV^* x \rangle = \langle x, x\rangle,
    \end{align*}
so if we set $y = V^*x$, it follows that 
\begin{align*}
    \max_{|x|_2 = 1} \langle \Sigma V^*x, V^*x\rangle &= \max_{|y_1|^2 + \hdots + |y_m|^2 = 1} \langle \Sigma_y, y \rangle \\
    &= \max_{|y_1|^2 + \hdots + |y_m|^2} \left(\lambda_1 |y_1|^2 + \lambda_2 |y_2|^2 + \hdots + \lambda_m |y_m|^2 \right) \\
    &= \max_j \lambda_j.
\end{align*}

From this, we can conclude that $||A||_2^2$ is the largest eigenvalue of $A^* A$. 
\end{proof}


\newpar
Suppose $A \in \C^{m \times m}$ and suppose that $A$ is diagonalizible. Then, $A = M D M^{-1}$ for some matrix $M$ and diagonal matrix $D$. If $A = A^*$, then $A$ is unitarily diagonalizible, so we can write $A = UDU^*$ for some unitary matrix $U$. Singular value decomposition generalizes this to all matrices $A \in \C^{n \times m}$.

\begin{definition}
    If $A \in \C^{n \times m}$, the \textit{Singular Value Decomposition (SVD)} of $A$ is of the form
        $$A = U \Sigma V^*$$
    where $V$ has dimensions $ m \times m$ and unitary, $\Sigma$ is diagonal and has dimensions $n times m$, and $U$ has dimensions $N \times n$ and is unitary.
\end{definition}

Assuming SVD exists, we can find the SVD by noting first that
    \begin{align}
        A^* A = M_1 D_1 M_1^*
    \end{align} 
where $M_1$ is unitary, $D_1$ is diagonal and has positive entries, and all matrices have dimensions $m \times m$. Furthemore, 
    \begin{align}
        AA^* = M_2 D_2 M_2^{-1},
    \end{align}
where $M_2$ is unitary and $D_2$ is diagonal and positive. By substituting $A = U \Sigma V^*$ into (1), we can conclude that 
    \[
        A^* A = V \Sigma^T U^* U \Sigma V^* = V \Sigma^T \Sigma V^* = M_1 D_1 M_1^*
    \]
Thus, if $\Sigma$ is diagonal with entries $\sigma_1, \hdots, \sigma_k$ with $K = \min{n, m}$, then $\sigma_j = \sqrt{\lambda_j}$ where $\lambda_j$ are eigenvaleus of $A^*A$. Furthermore, columns of $V$ are eigenvectors $v_1, \hdots, v_m$ of $A^* A$. To find $U$, we note that 
    \[
        Av_j = U \Sigma V^{-1} v_j = U \Sigma e_j = U \Sigma_j
    \]
where $e_j$ is the $j$-th standard basis vector and $\Sigma_j$ is the $j$-th column. Since $U \Sigma_j = \sigma_j U e_j = \sigma_j U_j$, it follows that $Av_j = \sigma_j U_j$. This completes the algorithm to find the SVD of a matrix $A$. 
\end{document}