% Packages and document formatting
\usepackage[margin=1in]{geometry} 
\usepackage{amsmath, amsthm, amssymb, amsfonts, fancyhdr, color, comment, graphicx, 
            environ, dsfont, centernot, mathtools, xcolor, mathrsfs, mdframed, physics}
\usepackage{tikz,tkz-euclide,tikz-cd}
\setlength{\parindent}{0 pt}

% Document structural environments
\global\mdfdefinestyle{problemstyle}{%
    linecolor = white, linewidth = 0.75pt, topline = true, bottomline = true, backgroundcolor = black!, nobreak = true, fontcolor = white
}
\newenvironment{problem}[2][Problem]
    {\begin{mdframed}[style=problemstyle] \textbf{#1 #2}}{\end{mdframed}}
\newenvironment{solution}{\begin{proof}[Solution]}{\end{proof}}
\newenvironment{solution*}{\begin{proof}[Solution]}{\end{proof}}
    \AtBeginEnvironment{solution*}{\renewcommand{\qedsymbol}{}}


\newenvironment{lemma}[2][Lemma.]
    {\begin{mdframed}[style=problemstyle] \textbf{#1 #2}}{\end{mdframed}}
\newenvironment{example}[2][Example.]
    {\begin{mdframed}[style=problemstyle] \textbf{#1 #2}}{\end{mdframed}}
\newenvironment{definition}[2][Definition.]
    {\begin{mdframed}[style=problemstyle] \textbf{#1 #2}}{\end{mdframed}}
\newenvironment{theorem}[2][Theorem.]
    {\begin{mdframed}[style=problemstyle] \textbf{#1 #2}}{\end{mdframed}}
\newenvironment{proposition}[2][Proposition.]
    {\begin{mdframed}[style=problemstyle] \textbf{#1 #2}}{\end{mdframed}}
\newenvironment{corollary}[2][Corollary.]
    {\begin{mdframed}[style=problemstyle] \textbf{#1 #2}}{\end{mdframed}}


\newcommand{\remark}{\textbf{Remark. }}
\newcommand{\defn}{\textbf{Definition. }}


% Formatting commands
\newcommand{\jump}{\vspace{0.25 in}}
\newcommand{\newpar}{\vspace{12 pt}}
\newcommand{\nsjump}{\vspace{-4 pt}}
\newcommand{\nljump}{\vspace{-8 pt}}
\newcommand{\tab}{\hspace*{0.25 in}}
\newcommand{\stab}{\hspace*{0.1 in}}


% Math operators
\DeclareMathOperator{\id}{id}
\DeclareMathOperator{\image}{im}
\DeclareMathOperator{\Imag}{Imag}
\DeclareMathOperator{\nullity}{nullity}
\DeclareMathOperator{\FS}{FS}
\DeclareMathOperator{\sla}{\mathfrak{sl}}
\DeclareMathOperator{\ipstar}{IP^\star}
\DeclareMathOperator{\ipstarlim}{IP^\star lim}
\DeclareMathOperator{\ud}{UD}
\DeclareMathOperator{\smod}{mod}
\DeclareMathOperator{\plim}{plim}
\DeclareMathOperator{\im}{Im}
\DeclareMathOperator{\re}{Re}
\DeclareMathOperator{\dist}{dist}
\DeclareMathOperator{\Area}{Area}
\DeclareMathOperator{\Int}{Int}
\DeclareMathOperator{\Ext}{Ext}
\DeclareMathOperator{\law}{law}


% Frequently used characters
\newcommand{\R}{\mathbb{R}}  
\newcommand{\Z}{\mathbb{Z}}
\newcommand{\N}{\mathbb{N}}
\newcommand{\Q}{\mathbb{Q}}
\newcommand{\C}{\mathbb{C}}
\newcommand{\F}{\mathbb{F}}

\newcommand{\vi}{\mathbf{i}}
\newcommand{\vj}{\mathbf{j}}
\newcommand{\vk}{\mathbf{k}}
\newcommand{\vx}{\mathbf{x}}
\newcommand{\vy}{\mathbf{y}}
\newcommand{\vz}{\mathbf{z}}
\newcommand{\vn}{\mathbf{n}}
\newcommand{\vr}{\mathbf{r}}
\newcommand{\vF}{\mathbf{F}}
\newcommand{\vG}{\mathbf{G}}
\newcommand{\vH}{\mathbf{H}}
\newcommand{\vE}{\mathbf{E}}
\newcommand{\vB}{\mathbf{B}}
\newcommand{\vP}{\mathbf{P}}
\newcommand{\vM}{\mathbf{M}}
\newcommand{\vA}{\mathbf{A}}
\newcommand{\vL}{\mathbf{L}}


\newcommand{\vell}{\boldsymbol{\ell}}
\newcommand{\vtheta}{\boldsymbol{\theta}}
\newcommand{\vphi}{\boldsymbol{\phi}}

\newcommand{\rhat}{\hat{\mathbf{r}}}
\newcommand{\phihat}{\hat{\boldsymbol{\phi}}}
\newcommand{\thetahat}{\hat{\boldsymbol{\theta}}}
\newcommand{\zhat}{\hat{\mathbf{z}}}
\newcommand{\yhat}{\hat{\mathbf{y}}}
\newcommand{\xhat}{\hat{\mathbf{x}}}
\newcommand{\nhat}{\hat{\mathbf{n}}}

\newcommand{\as}{\text{ as }}
\newcommand{\so}{\text{ so }}

\newcommand{\scrA}{\mathscr{A}}
\newcommand{\scrB}{\mathscr{B}}
\newcommand{\scrF}{\mathscr{F}}
\newcommand{\calL}{\mathcal{L}}


% New mathematical commands
\newcommand{\transpose}{\intercal}
\newcommand{\Deltastar}{\ensuremath{\Delta\hspace{-0.25ex}^\star}}
\newcommand{\poorDensity}[1]{\ensuremath{D_{\hspace{-0.3ex}#1}}}
\newcommand{\dbar}{\overline{d}}


% Quantum mechanics
\newcommand{\bilin}[2]{\langle #1 | #2 \rangle}
\newcommand{\sand}[3]{\langle #1 | #2 | #3 \rangle}
\newcommand{\expec}[1]{\left\langle #1 \right\rangle}
\newcommand{\vect}[1]{\overrightarrow{#1}}

% Miscellaneous commands
\renewcommand{\v}{\vb}
\renewcommand{\t}{\text}
\newcommand{\scr}{\mathscr}


% Include to be able to use \cupdot and \bigcupdot
\makeatletter
\def\moverlay{\mathpalette\mov@rlay}
\def\mov@rlay#1#2{\leavevmode\vtop{%
   \baselineskip\z@skip \lineskiplimit-\maxdimen
   \ialign{\hfil$\m@th#1##$\hfil\cr#2\crcr}}}
\newcommand{\charfusion}[3][\mathord]{
    #1{\ifx#1\mathop\vphantom{#2}\fi
        \mathpalette\mov@rlay{#2\cr#3}
      }
    \ifx#1\mathop\expandafter\displaylimits\fi}
\makeatother
\newcommand{\cupdot}{\charfusion[\mathbin]{\cup}{\cdot}}
\newcommand{\bigcupdot}{\charfusion[\mathop]{\bigcup}{\cdot}}


% Changed old commands
\renewcommand{\emptyset}{\text{\O}}
\newcommand{\oldepsilon}{\epsilon}
\renewcommand{\epsilon}{\varepsilon}
